\documentclass[a4paper,12pt]{article}
\usepackage[utf8]{inputenc}
\usepackage[polish]{babel}
\usepackage[T1]{fontenc}
\usepackage[dvipsnames]{xcolor}
\usepackage{songs}
\usepackage{geometry}
\geometry{margin=2.0cm}
\usepackage{multicol}
\usepackage[most]{tcolorbox}

\newtcolorbox{remarkbox}{
  colback=teal!10,
  colframe=gray,
  title=Uwaga,
  fonttitle=\bfseries,
}


\songcolumns{1} % Zmień na 2 dla dwóch kolumn
\begin{document}

\title{Mój Śpiewnik}
\author{Grzegorz}
\date{\today}
\maketitle

\tableofcontents
\newpage

\begin{songs}{}

\clearpage
\addcontentsline{toc}{section}{Bieszczadzkie anioły}
\beginsong{Bieszczadzkie anioły}[by={A. Ziemianin / K. Myszkowski}]
\begin{multicols}{2}
\begin{verse}
A\[a]nioły są takie \[G]ciche
\[G] Zwłaszcza te w Biesz\[G]czadach
Gdy \[a]spotkasz takiego w \[e]górach
\[e] Wiele z nim nie po\[e]gadasz

Naj\[C]wyżej na ucho ci \[G]powie
Gdy \[C]będzie w dobrym hu\[F]morze
Że \[C]skrzydła nosi w ple\[G]caku
\[a] Nawet przy \[e]dobrej po\[a]godzie
\end{verse}

\begin{verse}
A\[a]nioły są całe zie\[G]lone
\[G] Zwłaszcza te w Biesz\[G]czadach
\[a] Łatwo w trawie się \[e]kryją
\[e] I w opuszczonych \[e]sadach

W zie\[C]lone grają ukra\[G]dkiem
Nawet \[C]karty mają zie\[F]lone
Zie\[C]lone mają po\[G]jęcie
\[a] A nawet zie\[e]lony kie\[a]lonek
\end{verse}

\begin{chorus}
A\[C]nioły bie\[G]szczadzkie, bieszczadzkie \[a]anioły \[a]
Dużo w \[C]Was radości \[G] i dobrej po\[a]gody \[a]
Biesz\[C]czadzkie a\[G]nioły, anioły biesz\[a]czadzkie \[a]
Gdy \[C]skrzydłem Cię dotkną \[G] już jesteś ich \[a]bratem \[a]
\end{chorus}

\begin{verse}
A\[a]nioły są całkiem sa\[G]motne
\[G] Zwłaszcza te w Biesz\[G]czadach
\[a]W kapliczkach zimą \[e]drzemią
\[e]Choć może im nie wy\[e]pada

Czasem \[C]taki anioł sa\[G]motny
Za\[C]pomni dokąd ma \[F]lecieć
I \[C]wtedy całe Biesz\[G]czady
\[a] Mają sza\[e]loną u\[a]ciechę
\end{verse}

\begin{chorus}
Anioły bieszczadzkie... \rep{1}
\end{chorus}

\begin{verse}
A\[a]nioły są wiecznie u\[G]lotne
\[G] Zwłaszcza te w Biesz\[G]czadach
Nas \[a]też czasami \[e]nosi
\[e]Po ich anielskich \[e]śladach

\[C]One nam przyzwa\[G]lają
I \[C]skrzydłem wskazują \[F]drogę
I \[C]wtedy w nas się za\[G]pala
\[a] Wieczny \[e] bieszczadzki \[a]ogień
\end{verse}

\begin{chorus}
Anioły bieszczadzkie... \rep{2}
\end{chorus}

\end{multicols}
\endsong


\clearpage
\addcontentsline{toc}{section}{Dziś w nocy będzie fajnie}
\beginsong{Dziś w nocy będzie fajnie}[by={Maciej Zembaty — tłum. Leonarda Cohena}]

\begin{verse}
Czasem \[C]łapię się \[C]na tym, że się \[G]grzebię w przesz\[C]łości,
Przysię\[C]galiśmy sobie \[C] wytrwać w \[G]naszej mi\[C]łości,
Ty ko\[C]chałaś po\[C]woli — ja ska\[G]kałem jak \[C]pchełka,
Dziś ja \[C]jestem za \[C]cienki, a twa mi\[G]łość za \[C]wielka
\end{verse}

\begin{chorus}
Ale \[F]wiem \[F]— z oczu \[C]twych \[C]
uśmiech \[F]twój \[F] mówi \[C]mi, \[C]
Że tej \[F]nocy \[F] jeszcze \[C]raz... jeszcze \[C]raz
będzie \[G]nam fajnie \[G]przez jakiś \[C]czas. \[C]
\end{chorus}

\begin{verse}
Sam tros\[C]kliwie wy\[C]brałem to wspa\[G]niałe miesz\[C]kanie,
Okna \[C]są w nim za \[C]małe, nic nie \[G]wisi na \[C]ścianie,
Jedno \[C]tylko jest \[C]łóżko, tylko \[G]jedna mod\[C]litwa —
Aż do \[C]rana się \[C]modlę, żebyś \[G]jeszcze tu \[C]przyszła.
\end{verse}

\begin{chorus}
\rep{1}
\end{chorus}

\begin{verse}
Lubię \[C]patrzeć jak \[C]ona się \[G]dla mnie roz\[C]biera,
Naga \[C]mięknie od \[C]razu, gdy \[G]miłość w niej \[C]wzbiera,
Całym \[C]ciałem po\[C]rusza, jest o\[G]dważna i \[C]piękna —
Gdybym \[C]już nie za\[C]pomniał, chciałbym \[G]o tym pa\[C]miętać.
\end{verse}

\begin{chorus}
\rep{2}
\end{chorus}

\endsong

\clearpage
\begin{song}{Gdybym miał gitarę}[by=Ludowa]
\begin{remarkbox}
Metrum: 3/4
\end{remarkbox}

\begin{verse}
\[a]Gdybym miał gitarę
\[E7]To bym na niej \[a]grał
\[d]Opowiedziałbym \[a]o swej miłości
\[d]Którą \[E7]przeżyłem \[a/A7]sam. \rep{2}
\end{verse}

\begin{chorus}
\[a]A wszystko te czarne oczy
\[E7]Gdybym ja je \[a]miał
\[d]Za te czarne, \[a]cudne oczęta
\[d]Serce, \[E7]duszę bym \[a]dał.
\end{chorus}

\begin{verse}
\[a]Ludzie mówią głupi,
\[E7]Po coś ty ją \[a]brał
\[d]Po coś to dziewczę \[a]czarne, figlarne
\[d]Mocno \[E7]poko\[a/A7]chał. \rep{2}
\end{verse}

\begin{chorus}
\[a]A wszystko te czarne oczy...
\end{chorus}

\begin{verse}
\[a]Fajki ja nie palę,
\[E7]Wódki nie pi\[a]ję,
\[d]Ale z żalu, \[a]z żalu wielkeigo
\[d]Ledwo \[E7]co ży\[a/A7]ję. \rep{2}
\end{verse}

\begin{chorus}
\[a]A wszystko te czarne oczy...
\end{chorus}


\end{song}


\clearpage
\beginsong{Jak}[by={E. Stachura / K. Myszkowski}]
\vspace{-0.5cm}
\beginverse
\[D]Jak po nocnym niebie su\[A]nące białe o\[G]błoki nad \[D]lasem
Jak na \[e]szyi wędrowca a\[G]paszka szamotana \[D]wiatrem,
\endverse

\beginverse
\[D]Jak wyciągnięte \[A]tam powyżej gwiaź\[G]dziste ramiona \[D]wasze.
A \[e]tu są nasze, a \[G]tu są \[D]nasze.
\endverse

\beginverse
Jak \[D]suchy szloch w tę dż\[A]dżystą noc
Jak \[G]winny - li - niewinny su\[D]mienia wyrzut
Że się \[e]żyje, gdy u\[G]marło tylu, tylu, \[D]tylu
\endverse

\beginverse
Jak \[D]suchy szloch w tę dż\[A]dżystą noc
Jak \[G]lizać rany \[D]celnie zadane
Jak \[e]lepić serce w \[G]proch potrzas\[D]kane
\endverse

\beginverse
Jak \[D]suchy szloch w tę dż\[A]dżystą noc
Pu\[G]dowy kamień, pu\[D]dowy kamień
Ja \[e]na nim stanę, on \[G]na mnie stanie, On \[D]na mnie stanie, spod niego wstanę
\endverse

\beginverse
Jak \[D]suchy szloch w tę dż\[A]dżystą noc
Jak \[G]złota kula nad wo\[D]dami
Jak \[e]świt pod spuchnię\[G]tymi powie\[D]kami
\endverse

\beginverse
Jak \[D]zorze miłe, \[A]śliczne polany
Jak \[G]słońca pierś, jak \[D]garb swój nieść
Jak \[e]do was, siostry mgławi\[G]cowe, ten zawo\[D]dzący śpiew
\endverse

\beginchorus
Jak \[D]biec do końca, \[A]potem odpoczniesz, \[G]potem odpoczniesz
\[D]Cudne manowce, \[e]cudne manowce, \[G]cudne, cudne ma\[D]nowce \rep{3}
\endchorus
\endsong

\clearpage
\begin{song}{Mewy}[by=Andrzej Korycki]
\begin{multicols}{2}
\begin{verse}
\[a]Mewy białe \[F]mewy
\[G]Wiatrem rzeźbione \[a]z pian
Skrzy\[a]dlate białe \[F]muzy
O\[G]krętów odchodzących \[a]w dal
\[a]Kto wam szybować \[F]każe
\[G]Za horyzontu \[a]kres
W bez\[a]mierne oce\[F]any
Przez \[G(e)]sztormów świety \[a]gniew
\end{verse}

\begin{chorus}
Że\[F]glarzom wraca\[G]jącym \[a]z morza
Na \[F]pamięć przywo\[G]dzicie \[a]dom
Roz\[F]bitkom wasze \[G]skrzydła \[a]nio-so-\[F]ą
Na\[F]dzieję na \[G(e)]zbawienny \[a]ląd
\end{chorus}

\begin{verse*}
{\nolyrics \[a] \[F] \[G] \[a]}
{\nolyrics \[a] \[F] \[e] \[e]}
\end{verse*}


\begin{verse}
\[a]Ptaki zapamię\[F]tane
\[G]Jeszcze z dziecięcych \[a]lat
Dra\[a]pieżnie spada\[F]jące
Ze \[G]skał na szary Skager\[a]rak
Wiatr \[a]w grzywy czesał \[F]morze
Po \[G]falach skacząc lekko \[a]biegł
Pa\[a]miętam tamte \[F]mewy
Prze\[G]stworzy słonych \[a]zew
\end{verse}

\begin{chorus}
 Żeglarzom... \rep{3}
\end{chorus}

\newpage
\textit{(alternatywne chwyty refrenu)}
\begin{chorus}
Że\[F]glarzom wraca\[G]jącym \[a]z morza
Na \[F]pamięć przywo\[G]dzicie \[a]dom
Roz\[F]bitkom wasze \[G7]skrzydła \[C]nio-so-\[a]są
Na\[F]dzieję na \[E7]zbawienny \[a]ląd
\end{chorus}
\end{multicols}


\end{song}

\clearpage
\begin{song}{Od Turbacza}[by=Andrzej Mróz]
\begin{verse}
\[d]Od Turbacza wieje wiatr, \[A]niesie nam tę \[d]wieść
\[d]Że tej nocy szczyty gór \[A]pokrył biały \[d]śnieg
\[F]A w dolinach \[C]piękna jesień, \[d]złote liście \[A]lecą z drzew
\[d]Od Turbacza wieje wiatr, \[A]niesie zimny \[d]wiew
\end{verse}

\begin{chorus}
\[d]A -- \[G]a -- \[d]a -- a -- \[A]a -- a -- \[d]a
\end{chorus}

\begin{verse}
\[d]Zima białym płaszczem swym \[A]już okryła \[d]Tatry
\[d]Mgła zabrała słońcu blask, \[A]wieją zimne \[d]wiatry.
\[F]A w dolinach \[C]piękna jesień, \[d]złote liście \[A]lecą z drzew
\[d]Od Zawratu wieje wiatr, \[A]niesie zimny \[d]wiew.
\end{verse}


\begin{chorus}
\[d]A -- \[G]a -- \[d]a -- a -- \[A]a -- a -- \[d]a
\end{chorus}
\begin{verse}
\[d]Hej dziewczyno nie smuć się \[A]w ten jesienny \[d]czas
\[d]Chociaż raz uśmiechnij się, \[A]przywróć oczom \[d]blask.
\[F]To nic że na \[C]szczytach zima, \[d]a w dolinach \[A]jesień już
\[d]Uśmiech twój przemieni wszystko, \[A]wiosnę sercom \[d]wróci znów.
\end{verse}


\begin{chorus}
\[d]A -- \[G]a -- \[d]a -- a -- \[A]a -- a -- \[d]a
\end{chorus}
\end{song}


\clearpage
\begin{song}{Opadły mgły, wstaje nowy dzień}[by={E. Stachura / K. Myszkowski}]
\begin{multicols}{2}
\beginverse
Opadły \[C]mgły i miasto ze \[F]snu się budzi
\[C]Górą czmycha już \[G]noc
Ktoś tam \[C]cicho czeka, by \[F]ktoś powrócił
Do \[C]gwiazd jest bliżej niż \[G]krok
Pies się \[C]włóczy pod mu\[F]rami bezdomny
Niesie \[C]się tęsknota \[G]czyjaś na świata cztery \[C]strony
\hbox{}
A ziemia \[C]toczy, toczy swój \[F]garb uroczy
\[C]Toczy, toczy się \[G]los. \rep{2}
\hbox{}
\mbox{Ty, co \[C]płaczesz, ażeby \[F]śmiać mógł się ktoś}
Już \[C]dość! Już dość! Już \[G]dość!
Odpędź \[C]czarne myśli!
Dość już \[F]twoich łez
Niech to \[C]wszystko przepadnie we \[G]mgle.
\endverse

\beginchorus
Bo nowy \[C]dzień wstaje
Bo \[F]nowy dzień wstaje
\[C]No-owy \[G]dzień \rep{2}
\[C] \[F] \[C] \[G]
\endchorus

\beginverse
Z dusznego \[C]snu już miasto \[F]się wynurza
\[C]Słońce wschodzi gdzieś \[G]tam
Tramwaj \[C]na przystanku \[F]zakwitł jak róża
U\[C]chodzą cienie od \[G]bram
Ciągną \[C]swoje wózki dwu\[F]kółki mleczarze
Nad da\[C]chami snują się \[G]sny podlotków pełne \[C]marzeń
\hbox{}
A ziemia \[C]toczy, toczy swój \[F]garb uroczy
\[C]Toczy, toczy się \[G]los \rep{2}
\hbox{}
\mbox{Ty, co \[C]płaczesz, a żeby \[F]śmiać mógł się ktoś}
Już \[C]dość! Już dość! Już \[G]dość!
Odpędź \[C]czarne myśli
Porzuć \[F]błędny wzrok
Niech to \[C]wszystko zabierze już \[G]noc.
\endverse

\beginchorus
Bo nowy \[C]dzień wstaje
Bo \[F]nowy dzień wstaje
\[C]No-owy \[G]dzień \rep{4}
\endchorus
\end{multicols}
\end{song}


\begin{song}{Płonie ognisko}
\begin{multicols}{2}
\begin{verse}
Płonie \[a]ognisko \[E]i szumią \[a]knieje,
Druży\[E]nowy jest wśród \[a]nas.
Opo\[a]wiada staro\[E]dawne \[a]dzieje,
Boha\[E]terski wskrzesza \[a]czas. \[(G)]
\hbox{}
O ry\[C]cerstwie spod kresowych \[G]stanic,
O o\[d]brońcach \[E]naszych polskich \[a]granic,
A po\[a]nad nami \[E]wiatr szumny \[a]wieje,
I dę\[E]bowy huczy \[a]las. \rep{2}
\end{verse}

\begin{verse}
Już do \[a]odwrotu \[E]głos trąbki \[a]wzywa,
Alar\[E]mując ze wszech \[a]stron.
Wstaje \[a]wiara w or\[E]dynku szczę\[a]śliwa,
Serca \[E]biją w zgodny \[a]ton. \[(G)]
\hbox{}
Każda \[C]twarz się z uniesienia \[G]płoni,
Każdy \[d]laskę \[E]krzepko dzierży w \[a]dłoni,
A z mło\[a]dzieńczej się \[E]piersi wy\[a]rywa,
Pieśń po\[E]tężna pieśń jak \[a]dzwon.\rep{2}
\end{verse}

\begin{verse}
Płonie \[a]ogień jak \[E]serca go\[a]rący,
Rzuca \[E]w niebo iskry \[a]gwiazd.
Jedna \[a]przeszłość i \[E]przyszłość nas \[a]łączy,
Szumi \[E]wokół ciemny \[a]las. \[(G)]
\hbox{}
W blasku \[C]iskier jawi się his\[G]toria,
Tyle \[d]zdarzeń \[E]miało barwę \[a]ognia.
Przy o\[a]gnisku za\[E]siadły wspom\[a]nienia,
Dziejów \[E]kraju uczą \[a]nas. \rep{2}
\end{verse}

\begin{verse}
Gaśnie \[a]ognisko \[E]i szumią \[a]drzewa,
Spojrzyj \[E]weń ostatni \[a]raz.
Niech ci \[a]w duszy \[E]radośnie za\[a]śpiewa,
Że na \[E]zawsze łączą \[a]nas: \[(G)]
\hbox{}
Wspólne \[C]troski i radości \[G]życia,
Serc ha\[d]rcerskich \[E]zjednoczone \[a]bicia.
I ta \[a]przyjaźń \[E]najszczersza, praw\[a]dziwa
Którą \[E]Bóg połączył \[a]nas.\rep{2}
\end{verse}
\end{multicols}
\end{song}

\clearpage
\beginsong{Wędrówką życie jest człowieka}[by={E. Stachura / K. Myszkowski}]
\begin{verse*}
 \[G]A - \[D]a - \[e]a ... \[G] \[D] \[e] \[G] \[D] \[e] \[G] \[D] \[e]
\end{verse*}
%\begin{multicols}{2}
\begin{verse}
Wę\[e]drówką jedną życie jest czło\[G]wieka
Idzie \[D]wciąż, dalej \[e]wciąż,
Dokąd? \[G]Skąd? Dokąd? \[D]Skąd? Dokąd? \[e]Skąd?

Jak \[e]zjawa senna życie jest czło\[G]wieka;
Zjawia \[D]się, Dotknąć \[e]chcesz,
Lecz u\[G]cieka? Lecz u\[D]cieka! Lecz u\[e]cieka!
\end{verse}

\begin{chorus}
To \[C]nic! To \[G]nic! To \[D]nic!
D\[C]opóki sił jednak \[G]iść, przecież \[D]iść, będę \[e]iść!
To \[C]nic! To \[G]nic! To \[D]nic!
Do\[C]póki sił, będę \[G]szedł, będę \[D]biegł!
Nie dam \[e]się!
\end{chorus}

\begin{verse}
Wę\[e]drówką jedną życie jest czło\[G]wieka;
Idzie \[D]tam, Idzie \[e]tu,
Brak mu \[G]tchu? Brak mu \[D]tchu, brak mu \[e]tchu!

Jak \[e]chmura zwiewna życie jest czło\[G]wieka!
Płynie \[D]wzwyż, płynie w \[e]niż!
Śmierć go \[G]czeka? Śmierć go \[D]czeka, śmierć go \[e]czeka!
\end{verse}

\begin{chorus}
To nic!... \rep{3}
\end{chorus}
%\end{multicols}
\endsong

\beginsong{Za przełęczą}[by={Katarzyna Abramczyk}]
\begin{chorus}
W starej \[C]chacie \[D]malowany \[e]piec
Już nie\[C]długo \[D]będziesz z niego \[e]jeść
Za prze\[C]łęczą \[D]tkany zbożem \[e]dom
A cel \[C]drogi krętej \[D]tak daleko \[e]stąd
\end{chorus}

\begin{verse}
Czy pa\[e]miętasz wędrowcze, gdy \[h]idąc tam
Skromne \[C]chabry \[G]wyrzekły \[D]słowa:
\[e] „Skąd przychodzisz, do\[h]pokąd gnasz
\[C]Wita \[D]Cię noc ma\[e]jowa”
\end{verse}


\begin{chorus}
W starej chacie... \rep{1}
\end{chorus}

\begin{verse}
Czy pa\[e]miętasz, kiedy \[h]ślady złożyłeś
\[C]W dłoni lata \[G]rosy \[D]pełnej
Miarowej \[e]melodii kroków \[h]nie porzuciłeś
Na za\[C]wiłej drodze \[D]zawsze \[e]wiernej
\end{verse}

\begin{chorus}
W starej chacie... \rep{1}
\end{chorus}

\begin{verse}
Czy za\[e]chowasz w pamięci, gdy \[h]pójdziesz tam
Gdzie \[C]ogarnia tylko \[G]mgła bez\[D]senna
Gdy smętny \[e]wiatr Ci odpowie w i\[h]mieniu traw
„\[C]Żegna \[D]Cię noc je\[e]sienna”
\end{verse}

\begin{chorus}
W starej chacie... \rep{1}
\end{chorus}

\endsong
\clearpage

\clearpage
\begin{song}{Blowin' in the wind}[by={Bob Dylan}]
\begin{verse}
\[C]How many \[F]roads must a \[C]man walk \[a]down be\[C]fore you \[F]call him a \[C]man? \[G]
\[C]How many \[F]seas must a \[C]white dove \[a]sail be\[C]fore she \[F]sleeps in the \[G]sand?
Yes, 'n' \[C]how many \[F]times must the \[C]cannon balls \[a]fly be\[C]fore they're \[F]forever \[C]banned?
\end{verse}

\begin{chorus}
The \[d]answer, my \[G]friend, is \[C]blowin' in the \[F]wind,
The \[d]answer is \[G]blowin' in the \[C]wind.
\end{chorus}

\begin{verse}
\[C]How many \[F]years can a \[C]mountain e\[a]xist be\[C]fore it's \[F]washed to the \[C]sea? \[G]
Yes, 'n' \[C]how many \[F]years can some \[C]people e\[a]xist be\[C]fore they're a\[F]llowed to be \[G]free?
Yes, 'n' \[C]how many \[F]times can a \[C]man turn his \[a]head, and pre\[C]tend that he \[F]just doesn't \[C]see?
\end{verse}

\begin{chorus}
The \[d]answer, my friend... \rep{2}
\end{chorus}

\begin{verse}
Yes, 'n' \[C]how many \[F]times must a \[C]man look \[a]up be\[C]fore he can \[F]see the \[C]sky? \[G]
Yes, 'n' \[C]how many \[F]ears must one \[C]man \[a]have be\[C]fore he can \[F]hear people \[G]cry?
Yes, 'n' \[C]how many \[F]deaths will it \[C]take till he \[a]knows that \[C]too many \[F]people have \[C]died?
\end{verse}

\begin{chorus}
The \[d]answer, my friend... \rep{2}
\end{chorus}
\end{song}


\clearpage
\begin{song}{Odpowie ci wiatr}[by={Bob Dylan, tłum. Andrzej Bianusz}]
\begin{verse}
\[C]Przez ile \[F]dróg musi \[C]przejść każdy z \[a]nas, by \[C]mógł czło\[F]wiekiem się \[C]stać? \[G]
\[C]Przez ile \[F]mórz lecieć \[C]ma biały \[a]ptak, nim w \[C]końcu o\[F]padnie na \[G]piach?
\[C]Przez ile \[F]lat będzie \[C]kanion ten \[a]trwał, nim w \[C]końcu roz\[F]kruszy go \[C]czas?
\end{verse}

\begin{chorus}
Od\[d]powie Ci \[G]wiatr, wie\[C]jący przez \[F]świat,
Od\[d]powie Ci, \[G]bracie, tylko \[C]wiatr.
\end{chorus}

\begin{verse}
\[C]Przez ile \[F]lat będzie \[C]trwał góry \[a]szczyt, nim \[C]deszcz go na \[F]mórz zniesie \[C]dno? \[G]
\[C]Przez ile \[F]ksiąg pisze \[C]się ludzki \[a]byt, nim \[C]wolność wy\[F]pisze w nim \[G]ktoś?
\[C]Przez ile \[F]lat nie o\[C]dważy się \[a]nikt za\[C]wołać, że \[F]czas zmienić \[C]świat?
\end{verse}

\begin{chorus}
Od\[d]powie Ci \[G]wiatr, wie\[C]jący przez \[F]świat,
Od\[d]powie Ci, \[G]bracie, tylko \[C]wiatr.
\end{chorus}

\begin{verse}
\[C]Przez ile \[F]lat ludzie \[C]giąć będą \[a]kark nie \[C]wiedząc, że \[F]niebo jest \[C]tuż? \[G]
\[C]Przez ile \[F]łez, ile \[C]bólu i \[a]skarg przejść \[C]trzeba i \[F]przeszło się \[G]już?
\[C]Jak blisko \[F]śmierć musi \[C]przejść obok \[a]nas by \[C]człowiek zro\[F]zumiał swój \[C]los?
\end{verse}


\begin{chorus}
Od\[d]powie Ci \[G]wiatr, wie\[C]jący przez \[F]świat,
Od\[d]powie Ci, \[G]bracie, tylko \[C]wiatr.
Od\[d]powie Ci \[G]wiatr, wie\[C]jący przez \[F]świat,
I \[d]ty swą od\[G]powiedź rzuć na \[C]wiatr.
\end{chorus}
\end{song}

\clearpage
\begin{song}{Ukraina}
\begin{verse}
\[a]Hej tam, gdzieś znad Czarnej Wody \[E]wsiada na koń \[E7]Kozak młody,
\[a]Czule żegna się z dziewczyną, \[E7]jeszcze czulej z \[a]Ukra\[G]iną!
\end{verse}

\begin{chorus}
\[C]Hej! Hej! Hej sokoły! \[G]Omijajcie \[G7]góry, lasy, pola, doły!
\[a]Dzwoń! Dzwoń! Dzwoń dzwoneczku, \[E7]mój stepowy \[a]skowro\[G7]neczku!

\[C]Hej! Hej! Hej sokoły! \[G]Omijajcie \[G7]góry, lasy, pola, doły!
\[a]Dzwoń! Dzwoń! Dzwoń dzwoneczku, \[E7]mój stepowy \[a]dzwoń! \[E7]dzwoń! \[a]dzwoń!
\end{chorus}

\begin{verse}
\[a]Pięknych dziewcząt jest niemało, \[E]lecz najwięcej w \[E7]Ukrainie!
\[a]Tam me serce pozostało \[E7]przy kochanej \[a]mej dziew\[G]czynie!
\end{verse}

\begin{chorus}
\[C]Hej! Hej! Hej sokoły!...
\end{chorus}

\begin{verse}
\[a]Ona jedna tam została, \[E]jaskółeczka \[E7]moja-moja mała.
\[a]A ja tutaj w obcej stronie \[E7]dniem i nocą \[a]tęsknię \[G]do niej.
\end{verse}

\begin{chorus}
\[C]Hej! Hej! Hej sokoły!...
\end{chorus}

\begin{verse}
\[a]Żal, żal za dziewczyną! \[E]Za zieloną \[E7]Ukrainą!
\[a]Żal, żal, serce płacze, \[E7]już jej więcej \[a]nie zo\[G]baczę!
\end{verse}

\begin{chorus}
\[C]Hej! Hej! Hej sokoły!...
\end{chorus}

\begin{verse}
\[a]Wina, wina, wina, wina dajcie! \[E]A jak umrę po\[E7]chowajcie.
\[a]Na zielonej Ukrainie, \[E7]przy kochanej \[a]mej dziew\[G]czynie!
\end{verse}

\begin{chorus}
\[C]Hej! Hej! Hej sokoły!...
\end{chorus}
\end{song}

\clearpage
\begin{song}{Życie to nie teatr}[by={sł. E. Stachura / muz. J. Satanowski}]
\begin{small}
\begin{multicols}{2}
\begin{verse*}
\setlength{\leftskip}{0pt}
\setlength{\rightskip}{0pt}
1.
\[a]Życie to jest teatr, mówisz \[E]ciągle opowiadasz
\[E7]Maski coraz inne, coraz \[a]mylne się nakłada
\[F]Życie to zabawa, życie \[C]to jest jedna gra
\[G]Przy otwartych i zamkniętych \[C]drzwiach.
\hspace{1cm}To jest \[G]gra!

\[a]Życie to nie teatr, ja ci \[E]na to odpowiadam
\[E7]Życie to nie tylko kolo\[a]rowa maskarada
\[F]Życie jest straszniejsze i pię\[C]kniejsze jeszcze jest
\[E]Wszystko przy nim blednie,
\hspace{1cm} blednie \[a]nawet sama \[C7]śmierć!
\end{verse*}

\begin{chorus}
\[F]Ty i \[G]ja - teatry to są \[C]dwa. Ty i \[G]ja!
\[C]Ty - \[E7]ty prawdziwej nie uronisz \[a]łzy.
\[C7]Ty najwyżej w górę wznosisz \[F]brwi.
Nawet \[G]kiedy źle ci jest, to nie jest \[C]źle,
\hspace{1cm} bo ty \[G]grasz!

\[C]Ja - \[E7]duszę na ramieniu wiecznie \[a]mam
\[C7]Cały jestem zbudowany z \[F]ran
Lecz ka\[G]leką nie ja jestem tylko \[C]ty.
\hspace{1cm} Bo ty \[G]grasz!
\end{chorus}

\begin{verse*}
\setlength{\leftskip}{10pt}
\setlength{\rightskip}{0pt}
2.
Dzisiaj \[a]bankiet, u artystów, \[E]ty się tam wybierasz
\[E7]Gości będzie dużo, nieod\[a]stępna tyraliera
\[F]Flirt i alkohole, może \[C]tańce będą też,
Drzwi otwarte potem zamkną \[C]się.
\hspace{1cm} No i \[G]cześć!

\[a]Wpadnę tam na chwilę, zanim \[E]spuchnie atmosfera
\[E7]Wódki dwie wypiję, potem cicho się pozbieram
\[F]Wyjdę na ulicę, przy fon\[C]tannie zmoczę łeb
\[E]Wyjdę na przestworza,
\hspace{1cm} przecu\[a]downy stworzę \[G]wiersz.
\end{verse*}

\begin{chorus}

\[F]Ty i \[G]ja - teatry to są \[C]dwa. Ty i \[G]ja!
\[C]Ty - \[E7]ty prawdziwej nie uronisz \[a]łzy.
\[C7]Ty najwyżej w górę wznosisz \[F]brwi.
I nie\[G]zaraźliwy jest twój \[C]śmiech.
\hspace{1cm} Bo ty \[G]grasz!

\[C]Ja - \[E7]duszę na ramieniu wiecznie \[a]mam
\[C7]Cały jestem zbudowany z \[F]ran
Lecz gdy \[G]śmieję się to w krąg się śmieje \[C]świat.
\hspace{1cm} Cały \[G]świat!
\end{chorus}
%{\nolyrics \[C] \[E7] \[a] \[C7] \[F] \[G] \[C] \[G] \[C] \[E7]}

\end{multicols}
\end{small}
\end{song}

\end{songs}
\end{document}
